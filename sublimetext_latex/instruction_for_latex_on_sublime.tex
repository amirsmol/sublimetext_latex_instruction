\documentclass{memoir}
\usepackage{fullpage}
\usepackage{enumitem}

\begin{document} 

 
\section*{Installing \LaTeX sublime text on a windows or linux machine}
This instruction is prepared for someone who needs to produce high quality documents using \LaTeX.
He knows how to install Microsoft word on windows. 
He has access to Internet and knows how to use browser. 
He knows how to download files from Internet and how to run them.
He must have an administrative access to the computer he is installisublime text onto.
Computer literacy in general is needed. 
  
The software that you will need: Tex Live, sublime, latextools plugin. \\
Note: \LaTeX has two main distribution Tex Live and MiKTeX, we use Tex Live.  \\ 
Note: There are many editors for \LaTeX we use TeXlipse.  
\begin{enumerate}  
  \item Download and install \LaTeX  distribution
\begin{enumerate}
  \item Go to website https://www.tug.org/Tex Live/acquire-netinstall.html 
  \item Download and run install-tl-windows.exe
  \item Choose simple install (big) and click next
  \item Click install. This will unpack the installer into a temporary location. The installer will show up afterwards. Then click next
  \item Choose the mirror of your convenience from drop-down menu click next
  \item Choose the destination folder, remember this folder or write it down for future reference, then click next
  \item Choose the paper type that you will use for your documents then click next
  \item Check the installation setting or change them to your convenience and then click install
  \item Wait until installation completes this step will take some time depending on your system and Internet connection
\end{enumerate}
\item Download and instasublime text
\begin{enumerate}
  \item Go to http://wsublime text.com/3
  \item download the version related to your operating system
  \item install sublime text by running the downloaded file

\end{enumerate}
  \item Install the package control sublime text
  \begin{enumerate}
    \item goto https://packagecontrol.io/installation
    \item copy the text in the textbox and then go to the sublime text and then press "ctrl+`" this will open the console.
    \item paste the text in to the console.
    \item press enter
    \item restart sublime tex  
  \end{enumerate}
  
  \item install the latextools from list of packages in package manager
  \begin{enumerate}
    \item 
    \item select preference: package control: install package, a list will appear
    \item select latextools from the list, note the activities in the status bar in sublime text
    \item press "ctrl+shift+p", this will cause a list to open
  \end{enumerate}  
 

  \item Take care of of migrate preference error
  \begin{enumerate}
    \item Preferences: package settings: LaTeXTools: Reconfigure LaTeXTools and migrate settings
  \end{enumerate} 

  \item  Take care of COULD NOT COMPILE! error

  \begin{enumerate}
    \item  Click Sublime Text on menu bar, and then navigate to Preferences : Package Settings : LaTeXTools : Settings - User.
    \item  On around line 91, change builder from "traditional" to "simple".
  \end{enumerate} 

  \item  Set up the build system

  \begin{enumerate}
  \item Go to Tools->Build System in the sublime text menu
  \item Check LaTex in the menu (This will activate the latex build system)
  \item press "ctrl+b" and this will build your pdf file from the latex file
  \end{enumerate} 

  \item  Spell check 
  \begin{enumerate}
    \item  Try F6 in sublime text.
  \end{enumerate} 



  \item install sumatraPDF on windows (This will work only if you are on windows)
  \begin{enumerate}
    \item goto http://www.sumatrapdfreader.org/download-free-pdf-viewer.html
    \item download SumatraPDF-3.0-install.exe
    \item install SumatraPDF-3 by running the installation file=
  \end{enumerate}

  \item Set up SumatraPDF (This is optional, setting up LateTexTools with SumatraPDF will make the reviewing and editing easier)  

  \begin{enumerate}
  \item  Add the path to SumatraPDF to your path, (In order for LaTexTools to be able to recognize the SumatraPDF you should have sumatraPDF in your path)  
    \item Press the windows key, type environment variables
    \item Choose the path in your environment variables
    \item Add the path to Sumatra pdf after a ; to the other values of path
  \end{enumerate}



  \item Make the first Document
  \begin{enumerate}
    \item Run sublime text
    \item Choose file in the menu 
    \item Choose New File
    \item write your latex file document
    \item press "ctrl+b" to compile the file
  \end{enumerate}
  
  \item Finally, read, print, and review the document you have made with \LaTeX

\end{enumerate}

\end{document}
